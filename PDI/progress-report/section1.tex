\section{Introdução ao tema}\label{sec:intro}

% \begin{quote}
%  ``As atuais aplicações no domínio da automação industrial exigem cada vez mais comunicações robustas, fiáveis e determinísticas, 
%  em particular com requisitos de tempo-real. As soluções existentes neste domínio têm sido dominadas na última década por redes baseadas na tecnologia Ethernet,
%  em concreto as denominadas redes Ethernet Industrial e de Tempo-Real. Estas propostas caracterizam-se por extensões de tempo-real da proposta original da Ethernet 
%  como é o caso das soluções Ethercat, Profinet, EthernetIP, Sercos III, TT Ethernet, etc. 
%  
% A utilização deste tipo de redes é cada vez mais frequente em aplicações com exigências temporais criticas como é o caso da robótica, onde a grande maioria dos equipamentos atuais já incorporam este tipo de tecnologias.''
% \footnote{Excerto da descrição da proposta de dissertação fornecida pelo orientador}
% \end{quote}

\subsection{Contexto}\label{sec:contexto}

A recente digitalização da indústria introduz exigências cada vez maiores de comunicações robustas, fiáveis e determinísticas na área da automação industrial, em especial nas aplicações com requisitos de tempo-real.
Nos últimos anos tem-se assistido a uma crescente utilização de redes industriais baseadas na tecnologia \textit{Ethernet}. Geralmente denominadas redes de \textit{Ethernet} Industrial e de Tempo-Real, estas tecnologias são adaptações
da tecnologia \textit{Ethernet} original (IEEE 802.3) com o objetivo de interligar os dispositivos de um sistema com requisitos de temporais exigentes. As soluções \textit{EthernetIP}, \textit{Profinet} e \textit{EtherCAT} são os exemplos mais comuns
deste tipo de redes de comunicação.

A área da robótica industrial é um ótimo exemplo de que a utilização de redes \textit{Ethernet} de tempo-real faz sentido. A sincronização de ações entre componentes e a entrega e/ou difusão de mensagens cumprindo intervalos de tempo determinísticos são apenas duas
caraterísticas que tornam a utilização destas redes quase indispensável em sistemas robóticos modernos.

% - Redes industriais de comunicação em tempo real estão a dominar a industria;
% 
% - EtherCAT é uma dessas redes e contém um leque abrangente de possibilidades de utilização

\subsection{Motivação}\label{sec:motivacao}

A forte presença das redes \textit{Ethernet} de tempo-real na industria introduz a necessidade de dar formação acerca deste tema aos novos técnicos e engenheiros da área.
É, portanto, necessário colmatar a falta de formação acerca deste tema na oferta formativa do curso de Engenharia Eletrotécnica e de Computadores da FEUP\footnote{Faculdade de Engenharia da Universidade do Porto}.
A escassez no mercado de material didático de formação restringe o tipo de formação possível a bases puramente teóricas.
Assim pretende-se desenvolver um sistema de demonstração intuitivo capaz de evidênciar as capacidades e limitações destas redes, que será posteriormente incluído nas atividades letivas do curso.

Atendendo à atual disponíbilidade de material de redes EtherCAT presente na FEUP, decidiu-se utilizar esta tecnologia para o desenvolvimento do demonstrador em questão. 

% - Curso pretende adicionar à oferta formativa uma secção sobre redes de comunicacao industriais de tempo real;
% 
% - De modo a fornecer um entendimento mais fácil acerca das capacidades e limitações da rede \textit{EtherCAT}, faz sentido desenvolver um demostrador prático que possa ser utilizado para esse efeito;
%  - Porquê EtherCAT ??? -> Pq a feup ja tem esse material....
% 
% - Não existem muitos demostradores práticos e intuitivos no mercado para as demostrações pretendidas.


\subsection{Objetivos}\label{sec:objetivos}

O objetivo fundamental desta dissertação é o efetivo desenvolvimento de uma plataforma robótica de demonstração prática e intuitiva da rede \textit{EtherCAT} e da sua
respetiva aplicação de controlo.
Para isto, um estudo aprofundado da tecnologia e das funcionalidades deste protocolo é fundamental para que o sistema final possa apresentar um leque alargado
das suas capacidades e limitações.

Como o foco da demonstração é a utilização destas tecnologias em aplicações robóticas, a plataforma deverá ser baseada no controlo de trajetórias de eixos, com uma
configuração que permita, entre outros aspetos, demonstrar as caraterísticas da rede \textit{EtherCAT} ao nível da sincronização temporal (sincronização de ações) e da resposta
temporal (tráfego em tempo-real) da aplicação de controlo.

% \begin{quote}
%  ``Pretende-se nesta dissertação desenvolver um demonstrador que permita apresentar estas tecnologias, suportada numa aplicação industrial de elevado desempenho com requisitos temporais 
%  semelhantes aos existentes num manipulador robóticos. Este demonstrador será orientado para fins de ensino e será utilizado em diversas unidades curriculares do MIEEC.
% 
% O demonstrador será baseado numa rede Ethernet de Tempo-Real composta por vários nós do tipo Ethercat ou Profinet. Cada nó será baseado numa plataforma Raspberry PI 
% onde serão integrados módulos de comunicação NetHAT52 que suportam estes protocolos. Ao nível físico será utilizado um braço robótico, cujos motores (eixos) terão servo-controladores com interface de rede ethernet integrada.
% 
% Pretende-se desenvolver um conjunto de aplicações na área da robótica, em concreto controlo de trajetórias (velocidade e posição), que permitam demonstrar as características deste tipo de redes, 
% nomeadamente ao nível da sincronização temporal (sincronização de relógios) e da resposta temporal (suporte de tráfego tempo-real) das aplicações de controlo.''
% \footnote{Excerto da descrição da proposta de dissertação fornecida pelo orientador}
% \end{quote}
