\section{Trabalho realizado}\label{sec:trabalho}

\subsection{Interações com o orientador}\label{sec:interacoes}

\subsubsection{4 de novembro de 2020}
\begin{itemize}
    \item Apresentação do tema e dos objetivos da proposta de dissertação por parte do proponente;
    \item Apresentação do resultado esperado no fim do projeto (desenvolvimento efetivo da plataforma de demostração);
    \item Explicação dos objetivos e tarefas planeadas para ambos os semestres;
    \item Acordado entre orientador e mestrando manter ``reuniões'' regulares a cada, aproximadamente, duas semanas para efeitos
    de atualização do estado das tarefas e objetivos;
    \item Entrega de alguma documentação relevante ao mestrando, para que este possa iniciar o estudo necessário.
\end{itemize}

\subsubsection{25 de novembro de 2020}
\begin{itemize}
    \item Discussão acerca do presente relatório de progresso, onde foram esclarecidas algumas dúvidas acerca do conteúdo do mesmo,
    nomeadamente a diferença entre o contexto e a motivação do tema;
    \item Combinado um encontro presencial antes do fim das atividades letivas a 18 de dezembro de 2020 (sem data marcada), onde o 
    orientador fornecerá material ao mestrando para que este se possa ambientar ao \textit{hardware} e \textit{software} relevantes para o trabalho.
\end{itemize}



\subsection{Sequência de trabalhos até ao momento}\label{sec:seq_trabalhos}
\begin{enumerate}
    \item Aquisição de documentação adicional relativa ao protocolo \textit{EtherCAT}, em particular, a brochura
    de apresentação deste protocolo, distribuída pelo sítio \url{www.ethercat.org} em \url{www.ethercat.org/download/documents/ETG_Brochure_EN.pdf}
    \item Preparação de um repositório \textit{GitHub} para armazenamento e gestão de versões de código e documentação.
    \item Início do estudo do funcionamento da rede \textit{EtherCAT}, tendo até ao momento estudado:
        \begin{itemize}
            \item Formato dos pacotes de \textit{\textit{EtherCAT}}, inseridos numa trama \textit{Ethernet}
            \item Tipologias de rede suportadas
            \item Sincronização de relógios entre os dispositivos (\textit{Distributed Clocks})
        \end{itemize}
\end{enumerate}
