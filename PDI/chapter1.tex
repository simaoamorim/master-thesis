\chapter{Introdução} \label{chap:intro}

Este documento ilustra o formato a usar em dissertações na \Feup.
São dados exemplos de margens, cabeçalhos, títulos, paginação, estilos
de índices, etc. 
São ainda dados exemplos de formatação de citações, figuras e tabelas,
equações, referências cruzadas, lista de referências e índices.
Este documento não pretende exemplificar conteúdos a usar.

Uma recolha sobre as normas existentes pode ser encontrada em~\citet{kn:Mat93}.

Neste primeiro capítulo ilustra-se a utilização de citações e de
referências bibliográficas.

\section{Contexto} \label{sec:context}

Para além de dar um exemplo de utilização de uma citação, o parágrafo
seguinte introduz uma referência que pode ser consultada, entre muitas
outras referências bibliográficas interessantes~\citep{kn:Tha01,kn:PP05}.

\begin{quote}
  ``Like the Abstract, the Introduction should be written to engage the
  interest of the reader. It should also give the reader an idea of
  how the dissertation is structured, and in doing so, define the
  thread of the contents.''~\citep[chap.\ Introduction]{kn:Tha01} 
\end{quote}


\section{Motivação} \label{sec:motivation}

\blindtext


\section{Objetivos} \label{sec:goals}

\Blindtext




