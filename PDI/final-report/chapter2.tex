\chapter{Revisão Bibliográfica} \label{chap:sota}


\section{Definição e caraterização do problema}\label{sec:problem}

Com a modernização constante da indústria e consequente crescente exigência
de conetividade de todo o tipo de equipamentos, é imperativa a introdução
de novas soluções tecnológicas que satisfaçam estes requisitos modernos
mas que também mantenham a compatibilidade com as exigências satisfeitas
pelos sistemas atualmente em uso.

Quando nos referimos à área da robótica, exigências temporais como o
período, a latência e a periodicidade da malha de controlo são fulcrais
para o funcionamento estável do sistema. Ora, a crescente exigência de
conetividade moderna instiga à utilização de redes de comunicação no
ambiente industrial. Quando o controlo de diversas áreas num equipamento
é feito de modo central num PLC ou micro-controlador, a sincronização de
alteração de estados de saídas é trivial, pois basta garantir que ambas
são atualizadas no mesmo ciclo de processamento. Quando se pretende
introduzir uma rede de comunicação entre o processamento central e o
controlo das saídas e/ou a aquisição das entradas, é crucial garantir
que não existem atrasos significativos na transmissão da informação
na rede e garantir a sincronização entre as atualizações das saídas
nos diversos escravos.

Nos últimos anos, várias implementações e estudos já foram realizados
com manipuladores robóticos utilizando redes de comunicação na malha de
controlo como \cite{Zhang18}, \cite{leiwang2010} ou muito recentemente
\cite{deremetz2020}. No entanto, todas elas se focam na vertente mais
industrial e técnica da solução e praticamente não existem implementações
focadas no ensino.

% económico
Um demonstrador focado no ensino deve ter caraterísticas apelativas de
modo a que a sua utilização possa ser o mais generalizada possível. O
primeiro objetivo deverá ser o desenvolvimento de um produto económico
de modo a que este possa ser adquirido em larga escala pelos
estabelecimentos de ensino. É também preciso considerar que no ambiente
de ensino, é mais provável acontecerem situações de utilização indevida
de um equipamento do que em ambiente industrial, onde geralmente apenas
pessoas qualificadas estão autorizadas a interagir com o mesmo, e portanto
uma possível avaria do produto não pode trazer prejuízos avultados à
instituição e/ou ao estudante. Naturalmente, esta redução no custo
implicará sempre uma redução na qualidade do produto final face a um
desenvolvimento de nível industrial, mas essa não é uma caraterística
fundamental de um demonstrador didático.

% implementação simples
Muitas vezes o momento em que nos é apresentada uma tecnologia desconhecida
através de um demostrador didático surgem questões acerca do demostrador
propriamente dito e não na tecnologia que ele pretende demonstrar. Assim,
para minimizar este tipo de interferência na aprendizagem existem duas
possíveis soluções: utilizar uma ideia de base que seja tão simples que
não permita qualquer tipo de dúvida ou que o público alvo tenha um
conhecimento aprofundado da ideia de base. Considerando que o público alvo
deste demostrador são estudantes de mestrado na área da robótica,
o controlo de movimento e posição de um manipulador robótico são temas
bem conhecidos.

% componentes 'off the shelf'
Complementando a caraterística económica de um demonstrador educativo,
a utilização de componentes e sub-sistemas genéricos, denominados
componentes \emph{off-the-shelf}, facilita a aquisição dos mesmos tanto
para o fabrico como para reparações. Assim, é possível que o próprio
cliente faça uma reparação do produto, sendo que a ação mais simples será
a troca do componente ou sub-sistema danificado.

% modularidade
A utilização de componentes e sub-sistemas genéricos leva-nos a um ponto
importante de demonstradores educativos: a modularidade. Um sistema
dividido em sub-sistemas mais simples que se focam numa única tarefa é um
sistema modular. Cada sub-sistema por si é simples, fácil de implementar,
interpretar e diagnosticar. A conjugação dos diferentes sub-sistemas,
encadeados e inter-ligados permite obter um sistema mais complexo com a
vantagem de que o seu desenvolvimento e eventual diagnóstico possa ser
feito por partes, o que simplifica tal processo, proporcionando também a
possibilidade de este ser paralelizado. Esta modularidade permite que,
no contexto de aprendizagem, seja mais fácil e rápido interpretar o
funcionamento de cada sub-sistema e, consequentemente, interpretar o
funcionamento geral do sistema.

% intuitivo
Por fim, a caraterística mais importante de qualquer demonstrador
educativo, e razão pela qual estes existem, é o seu caráter intuitivo.
Qualquer utilizador tem de ser capaz de, através do próprio funcionamento
do demonstrador, entender o conceito base em exposição. Em ambiente
educacional é muito importante fornecer este tipo de contacto com a
tecnologia para que os estudantes possam sedimentar os conhecimentos com
mais facilidade e de uma forma mais duradoura. % TODO arranjar citação para isto


\section{\ecat}\label{sec:ethercat}

A rede \ecat é uma rede de \emph{Ethernet} industrial que usa a
especificação padrão IEEE 802.3 \cite[]{ieee:IEEEStandardEthernet} para
definir o formato dos \emph{frames} e camada física a utilizar, mas
introduz uma maneira diferente de os processar.

Esta nova forma de processamento permite uma comunicação com todos os
dispositivos presentes na rede com apenas um \emph{frame}. \ecat utiliza
uma tipologia de comunicação \emph{Master/Slave}, tipicamente implementada
numa arquitetura de rede encadeada (\emph{daisy-chain}), mas permite várias
outras arquiteturas %\cite{}. % TODO: Citar aqui algo que explicite as varias arquiteturas

\subsection{Arquiteture de rede encadeada}
Apenas o dispositivo mestre pode iniciar um \emph{frame} de comunicação
e os dispositivos escravos limitam-se a ler a parte informação contida
no \emph{frame} que lhes é endereçada. Ao mesmo tempo, cada dispositivo
escravo pode introduzir informação sua  no \emph{frame} antes de o enviar
para o dispositivo seguinte.

\subsection{sincronização de relógios}


\subsection{Conclusão}
Estas caraterísticas permitem que o dispositivo mestre seja implementado
em qualquer tipo de dispositivo que contenha uma porta de comunicação 
\emph{Ethernet}. Os dispositivos escravo utilizam um \emph{EtherCAT Slave
Controller} (ESC) que processa os \emph{frames} fazendo com que a velocidade
e tempos de resposta da rede sejam previsíveis e independentes dos 
dispositivos escravo que existam na rede. Assim, é possível a utilização
de dispositivos escravo implementados em arquiteturas de computação
diferentes dentro da mesma rede \ecat.


\section{Soluções propostas} \label{sec:solution}

Para atingir os objetivos propostos por esta dissertação, foram propostas 
duas possíveis soluções. Ambas são apresentadas de seguida sendo que
maior ênfase será dada na última, pois é a proposta que se mostra mais
adequada ao estudo em questão.

Ambas as soluções têm por base o controlo de movimento através da velocidade
e/ou posição de um sistema robótico de múltiplos eixos. Fazendo uso de
uma arquitetura de controlo distribuída interligada por uma rede \ecat
em tipologia encadeada (secção \ref{sec:daisychain}). Esta arquitetura
será constituída por um dispositivo mestre implementado num micro-computador
\emph{Raspberry Pi}, programado através das linguagens descritas no padrão
IEC 61161-3. Os dispositivos escravo, que farão a interface com os atuadores,
sensores e interface de comando, serão implementados através de placas
\emph{Arduino UNO} agrupado com o adaptador \emph{EasyCAT} da \cite{ABT:EasyCAT}.
Um esquema da arquitetura pretendida é mostrado na figura
\ref{fig:network-architecture}.

\begin{figure}
 \centering
 \includegraphics[width=0.75\linewidth]{network-diagram_transparent.png}
 \caption{Arquitetura da rede \ecat pretendida}
 \label{fig:network-architecture}
\end{figure}



\subsection{Controlo de discos perfurados}

A primeira solução idealizada no contexto desta dissertação baseia-se num
sistema constituído por discos perfurados na sua extremidade, acoplados
a motores rotativos independentes. Por sua vez estes motores serão
controlados por módulos \ecat escravo interligados numa configuração
encadeada (\emph{daisy-chain}), tendo um \emph{Raspberry Pi} como dispositivo
mestre.

Cada dispositivo \ecat escravo será composto por um Arduino UNO 
\cite[]{arduino:ArduinoUNORev3} e um \emph{shield EasyCAT} da 
\cite{ABT:EasyCAT}. O dispositivo mestre, implementado num \emph{Raspberry
pi}, será programado com o ambiente de desenvolvimento industrial \emph{
Codesys}.

\subsection{Seguimento de um traçado com um braço robótico}


