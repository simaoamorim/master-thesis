\chapter{Plano de trabalho}\label{chap:chap3}

Neste capítulo apresenta-se o planeamento de tarefas, seus objetivos,
faseamento e metodologia de abordagem. No final será feita uma breve
apresentação das tecnologias e ferramentas escolhidas que permitirão
realizar as tarefas e cumprir os objetivos propostos.


\section{Tarefas, objetivos e metodologia de abordagem}

Para poder iniciar os trabalhos com uma rede \ecat\, será necessário,
naturalmente, ter uma rede provisória funcional onde se possa iniciar o
desenvolvimento do software de controlo. Para isto, está planeada a
montagem de uma pequena rede \ecat\ através de um \raspi\ como dispositivo
mestre e, no mínimo, dois conjuntos \arduino\ + \emph{EasyCAT} como
dispositivos escravo, cada um com um motor e codificador de posição
associados. Esta rede de prototipagem permitirá o desenvolvimento de uma
parte substancial do código de controlo, tanto do lado do \raspi\ como
do lado dos \arduino.

\subsection{Calendarização}
% TODO incluir grafico gantt
De forma a tornar mais percetível a calendarização de tarefas, criou-se
um gráfico do tipo \emph{Gantt} para a ilustrar.


\subsection{Tecnologias e ferramentas a usar}
Como descrito na secção \ref{sec:solution}, ir-se-á utilizar diversos
equipamentos para satisfazer requisitos diferentes, como por exemplo o
\raspi \cite[]{foundation:RaspberryPi} e dos \arduino. O primeiro será programado utilizando a plataforma 
\codesys\ \cite[]{CODESYS:codesys} que permite a utilização das linguagens
padrão da indústria de automação, IEC 61161-3. Os \arduino\ serão programados
em liguagem C, utilizando as librarias e compilador dedicados para os
micro-controladores AVR, que o \arduino\ utiliza. Para que este possa
funcionar como um escravo de \ecat, utilizar-se-á um adaptador \easycat\
da \cite{ABT:EasyCAT} e as respetivas librarias para desenvolver um
programa de controlo capaz de comandar a velocidade de rotação do motor
e de fazer a contagem de impulsos provenientes do codificador de posição.
O valor desta contagem deverá ser enviado para o \raspi\ de forma a que
este possa fazer o controlo de posição dos motores.

