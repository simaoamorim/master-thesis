\chapter{Introdução} \label{chap:intro}




\section{Contexto}\label{sec:contexto}

A recente digitalização da indústria introduz exigências cada vez maiores de comunicações robustas, fiáveis e determinísticas na área da automação industrial, em especial nas aplicações com requisitos de tempo-real.
Nos últimos anos tem-se assistido a uma crescente utilização de redes industriais baseadas na tecnologia \textit{Ethernet}. Geralmente denominadas redes de \textit{Ethernet} Industrial e de Tempo-Real, estas tecnologias são adaptações
da tecnologia \textit{Ethernet} original (IEEE 802.3) com o objetivo de interligar os dispositivos de um sistema com requisitos de temporais exigentes. As soluções \textit{EthernetIP}, \textit{Profinet} e \textit{EtherCAT} são os exemplos mais comuns
deste tipo de redes de comunicação.

A área da robótica industrial é um ótimo exemplo de que a utilização de redes \textit{Ethernet} de tempo-real faz sentido. A sincronização de ações entre componentes e a entrega e/ou difusão de mensagens cumprindo intervalos de tempo determinísticos são apenas duas
caraterísticas que tornam a utilização destas redes quase indispensável em sistemas robóticos modernos.


\section{Motivação}\label{sec:motivacao}

A forte presença das redes \textit{Ethernet} de tempo-real na industria introduz a necessidade de dar formação acerca deste tema aos novos técnicos e engenheiros da área.
É, portanto, necessário colmatar a falta de formação acerca deste tema na oferta formativa do curso de Engenharia Eletrotécnica e de Computadores da FEUP\footnote{Faculdade de Engenharia da Universidade do Porto}.
A escassez no mercado de material didático de formação restringe o tipo de formação possível a bases puramente teóricas.
Assim pretende-se desenvolver um sistema de demonstração intuitivo capaz de evidênciar as capacidades e limitações destas redes, que será posteriormente incluído nas atividades letivas do curso.

Atendendo à atual disponíbilidade de material de redes EtherCAT presente na FEUP, decidiu-se utilizar esta tecnologia para o desenvolvimento do demonstrador em questão. 


\section{Objetivos}\label{sec:objetivos}

O objetivo fundamental desta dissertação é o efetivo desenvolvimento de uma plataforma robótica de demonstração prática e intuitiva da rede \textit{EtherCAT} e da sua
respetiva aplicação de controlo.
Para isto, um estudo aprofundado da tecnologia e das funcionalidades deste protocolo é fundamental para que o sistema final possa apresentar um leque alargado
das suas capacidades e limitações.

Como o foco da demonstração é a utilização destas tecnologias em aplicações robóticas, a plataforma deverá ser baseada no controlo de trajetórias de eixos, com uma
configuração que permita, entre outros aspetos, demonstrar as caraterísticas da rede \textit{EtherCAT} ao nível da sincronização temporal (sincronização de ações) e da resposta
temporal (tráfego em tempo-real) da aplicação de controlo.
