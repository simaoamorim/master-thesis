\documentclass[10pt,a4paper,draft]{article}
\usepackage[british]{babel}
\usepackage[utf8]{inputenc}
\usepackage{blindtext}
\usepackage{csquotes}

% Watermark
% Comment out when releasing
\usepackage{draftwatermark}
\SetWatermarkScale{4}

% Custom header
%\usepackage{fancyhdr}
%\pagestyle{fancy}
%\fancyhf{}
%\cfoot{\thepage}
%\chead{\copyright\ Simão Amorim, 2021}
%\renewcommand{\headrulewidth}{0pt}
%\renewcommand{\footrulewidth}{0pt}

\title{EtherCAT demonstrator - User Manual}
\author{Simão Amorim}

\begin{document}
	\maketitle
%	\thispagestyle{fancy}
	
%	\begin{abstract}
%		\blindtext
%	\end{abstract}
	
	\section{Introduction}
	\blindtext
	
	\section{Exercises}
	Do not advance through the stages without making sure you noted every aspect of the simulations: there is
	no going back to the previous stage and, because every simulated fault is picked randomly, it's possible only

	\subsection{Scenario selection}
	\begin{enumerate}
		\item Choose a pattern scenario. If you don't want to choose one, select the \enquote{random} option.
		\item Watch the default behaviour of the system. Take notes about the system performance and make
		sure you know how every action that is happening. You can repeat the process you think you missed
		something.
	\end{enumerate}
	\subsection{Fault identification}
	\begin{enumerate}
		\item Advance to the next stage on the demonstrator. It has chosen to simulate some random fault.
		\item Run the pattern again and pay attention to the system behaviour. The scenario chosen in the first 
		stage is now running with a simulated network fault. Note any differences with the previous run. (Tip:
		some faults need a specific situation in order to be relevant. Run experiments using the available controls
		and try to figure out when the effect is relevant)
		\item Describe the observed behaviour and try to identify what type of communication fault could originate
		such behaviour, explaining your thought process.
	\end{enumerate}
	\subsection{Confirmation}
	\begin{enumerate}
		\item Finalize the simulation by advancing to the final stage and note what fault the system simulated.
		\item What is the rationale behind this particular simulation? Why did the system behave differently in
		the presence of this fault?
		\item Did you identify all the relevant simulation changes?
		\item Make a comparison about the identification you performed on the second stage and explain why
		it was correct, incorrect or incomplete, whichever the case.
	\end{enumerate}
	
	\section{Closing thoughts}
	\blindtext
		
\end{document}