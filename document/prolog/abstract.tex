\chapter*{Abstract}
%\addcontentsline{toc}{chapter}{Abstract}
We live in an increasingly digital and computerized world where there is a constant need for interconnection between everything and everyone.
Ethernet networks quickly became the communication stardard in office and home environments, but their adoption in the industrial environment has been much slower.
This reduced speed of adoption is mainly due to the non-deterministic communication that Ethernet network provides, which does not make it a viable option for automation and robotics systems that require predictable and deterministic communications.
Modern automation and robotic systems do not escape the accelerated need for constant interconnection and, therefore, it is necessary to adapt them, taking into account their real-time requirements.
There are several well-established real-time Ethernet network solutions on the market, but we find the same gap on all of them: the scarcity of educational and demonstration equipment.

This document aims at presenting a solution for this gap, describing a distributed control system developed with education in mind.
The presented system focuses on the variable that mostly contributes to the deterministic characteristic of real-time Ethernet networks: the network cycle time.
The main objective is to develop a conceptual distributed control system capable of producing experimental data that demonstrates the impact that the communication network's cycle time has on control applications.
The proposed system will base itself on the implementation of a slave device for the EtherCAT network, implemented on a Raspberry Pi platform, and on the description of a possible implementation of a master device for the same network.
The implementation of the master device will purposely not be specified in detail to encourage greater involvement from the users of this system in implementing their own educational or demonstration system.

The document will begin with a presentation of the context in which this work was developed, followed by the motivation that led to the beginning of the project and a presentation of the objectives that we propose to achieve.
The state-of-the-art will consist of a presentation of real-time Ethernet networks in a generic way and an in-depth presentation of the working principle and characteristics of the EtherCAT network.
The presentation of generic real-time Ethernet networks will describe the existing categories and different approaches to the problem of non-deterministic communication in Ethernet networks.
The presentation and description of the architecture of the proposed distributed control system will follow.
Next, an explanation on how the implementation of the slave device was planned and executed, both in terms of hardware and software.
Finally, experimental results will also be presented.
These prove that the developed concept is valid and fulfills the intended characteristics. 


\chapter*{Resumo}
%\addcontentsline{toc}{chapter}{Resumo}
Vivemos num mundo cada vez mais digital e informatizado onde existe uma constante necessidade de interligação entre tudo e todos.
As redes Ethernet rápidamente se tornaram no stardard da comunicação em ambientes empresariais e domésticos, mas a sua adoção no ambiente industrial tem sido bastante mais lenta.
Esta reduzida velocidade de adoção deve-se principalmente à comunicação não-determinística que a rede Ethernet proporciona, o que não a torna uma opção viável para sistemas de automação e robótica que necessitão de comunicações previsíveis e determinísticas.
Os automatismos e sistemas robóticos modernos não escapam à acelerada necessidade de constante interligação e, por isso, é necessário adaptá-los tendo em consideração os seu requisitos de tempo-real.
Existem no mercado várias soluções de redes Ethernet de comunicação de tempo real, já bem estabelecidas, mas em todas se encontra a mesma lacuna: a escassez de equipamento educativo e de demostração das mesmas.

O presente documento pretende apresentar uma solução para tal lacuna, descrevendo um sistema baseado em controlo distribuido desenvolvido a pensar na educação.
O sistema apresentado foca-se na variável com maior impacto na característica determinística das redes de Ethernet de tempo-real: o tempo de ciclo da rede.
O objectivo principal é desenvolver um conceito de um sistema de controlo distribuído capaz de produzir dados experimentais que demonstrem o impacto que o tempo de ciclo da rede de comunicação tem em aplicações de controlo.
O sistema proposto será baseado na implementação de um dispositivo escravo para a rede EtherCAT, implementado numa plataforma Raspberry Pi e na descrição de uma possível implementação de um dispositivo mestre para a mesma rede.
A implementação do dispositivo mestre será propositadamente deixada em aberto para incitar um maior envolvimento dos utilizadores deste sistema na implementação do seu próprio sistema educacional ou de demonstração.

O documento iniciará com uma apresentação do contexto em que este trabalho foi desenvolvido, seguido da motivação que levou à realização do projeto e da apresentação dos objetivos que propomos cumprir.
A revisão bibliográfica consistirá na apresentação de redes Ethernet de tempo-real de uma forma genérica e na apresentação aprofundada do modo de funcionamento e características da rede EtherCAT.
A apresentação de redes Ethernet de tempo-real genérica irá descrever as várias categorias existentes e as diferentes abordagens ao problema da falta de determinismo na comunicação Ethernet.
Seguir-se-á a apresentaçao e descrição da arquitetura do sistema de controlo distribuído proposto.
Seguidamente será explicado como foi planeada e executada a implementação do dispositivo escravo, tanto em termos de hardware como software.
Finalmente também irão ser apresentados resultados experimentais que provam que o conceito desenvolvido é válido e cumpre com as características pretendidas.

