\section{Parts choice} \label{sec:parts_choice}


% Parts choice
% `- Motor & Encoder
% `- DFR0592
% `- Raspberry Pi
% `- netHAT 52-RTE
% `- screw terminal add-on

\subsubsection{Raspberry Pi 4}
The Raspberry Pi 4 is a single board computer (SBC) and comes equiped with the Broadcom's BCM2711, a quad-core Cortex-A72 64-bit ARM processor clocked at 1.5GHz \cite{technology:rpi4-specs}.
At the time of writting, versions were available with 2, 4 and 8GB of LPDDR4 SDRAM \footnote{Low-Power Double Data Rate Synchronous Dynamic Random-Access Memory} clocked at 3200MHz.

This version of the Raspberry Pi series is the first to be equipped with a true-Gigabit Ethernet controller connected to the PCIe bus, while earlier versions used a USB attached one, meaning latency and throughput were not as good and especially less constant.

As our designed slave device is inteded to be used in headless mode, meaning no monitor output and no keyboard nor mouse will be used, we picked the version with 2GB of RAM.
As no graphical interface needs to be created, memory usage will be very reduced and, as such, 2GB are plenty of memory for our needs.

The Rasbperry Pi 4 incorporates a microSD card slot to be used as an embedded hard disk, so we have also included a small 16GB microSD card to serve as such.
Linux is a very small operating system and a fresh install of Raspberry Pi OS Lite occupies about 1.4GB, meaning the 16GB are also more than sufficient for our needs.

\subsubsection{Motor \& encoder}


\subsubsection{DFRobot's DFR0592}


\subsubsection{Hilscher's netHAT 52-RTE}


\subsubsection{Screw terminal GPIO interface}

