\section{Proposed implementation}
Following the proposed architecture, presented in the previous chapter, we will now explain the actual implementation we performed to achieve our goals.
This section will briefly present the overall idea and proposed system architecture without going into details about specific choices.
Those explanaitions will be presented in later sections, as we will delve deeper into the implementation details, including product and technology choices, both in terms of hardware and software.

\subsection{Master node}
The master node of our system will be implemented in a generic desktop PC through the usage of an industrial programming platform.
This will enable us to program the behaviour of the master node as well as provide the necessary communication libraries to implement a master node for several RTE networks.
We will take advantage of the fact that most industrial programming platforms allow us to determine the RTE network's update period.
This will enable us to perform tests using different network cycle times.

Two programs will be developed for the master node:
\begin{enumerate}
	\item one to act as a simple set-point generator, sending the velocity or position set-points to the slave device throught the RTE network;
	\item a second one to act as the motion controller, where the same set-points are used internally in a control algorithm that receives the plant feedback value and sends the plant output value through the RTE network.
\end{enumerate}

This implementation will allow us to perform the two practical experiments described in \autoref{sec:experiments}, enabling us to compare performance values acquired in both cases.

\subsection{Slave node}
The proposed slave implementation will be based on an embedded computing platform.
The embedded platform will be extended using some specialised boards that will broaden the functionallity of the slave device as a whole.
These extension will provide easier access to the GPIO pins, direct interface with a motor through a specialized DC motor control board and Real-Time Ethernet connection using a dedicated board capable of off-loading the real-time processing of network packets from the embedded computing platform.

In terms of software, a control application will be developed for the slave device computing platform in order to provide the following functionality on the slave device:
\begin{itemize}
	\item Handle the receiving and sending of cyclic data through the RTE network by interfacing with the driver of the dedicated RTE network connection board. Such data will include set-point and feedback values;
	\item Handle the plant feedback signals, converting them into internal variables;
	\item Handle the motor output signal by interfacing with the dedicated DC motor control board;
	\item Acquire and export performance data relating to the control of the DC motor speed and/or position;
	\item Provide an internal control algorithm to locally control the motor's speed and/or position;
\end{itemize}
