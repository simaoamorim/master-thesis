\section{Requirements analysis} \label{sec:requirements}

% Simplicity
% Low cost
% Robust (SW & HW)
% Modularity
% Visually/Physically based
% Control:
% - DCS
% - Local & Remote control
% - Velocity & Position

When considering the development of any practical demonstrator, some generic requirements should be taken into consideration due to the scope of such products.
These become even more relevant when the demonstrator is designed for educational purposes, as this is one such scenario.

The following subsections will delve into the main requirements of our project, explaining why they are being considered a requirement, how they were dealt with during development phase, what difficulties were encountered to meet such need and in which manner each requirement influenced the final system.

\subsection{Simplicity}

The most important characteristic every demonstrator should own is simplicity.
No matter how complex or extensive the concept might be, good demonstrators are conceptually simple.
Designs that focus solely on the concept at hand and leave out superfluous functionality tend to be more effective at conveying the main message.
Having the ability to further explore the concept beyond the initial scope of the demonstrator by extending its capabilities could be an advantage, but only when the implications of doing so do not hurt the initial simplicity.

In order to design a good demonstrator, simplicity should be the main focus in the early stages of research and concept design.
Contemplating different approaches based on simple concepts is crucial to ensure the end result is focused on the correct concept.
It is also very important to not allow underlying characteristics or design choices to outweigh the core concept.

\subsection{Low-cost}

Demonstrators whose purpose is to serve as a first contact mechanism under an educational directive should be as low-cost as possible.
Accidents, bad practices or the simple lack of necessary knowledge can lead students and first-timers to, unintentionally, damage educational equipment.

Students tend to learn more easily when left to their own experiments, learning by themselves how things work and how to operate them. % TODO Cite something
For this to be possible, educational equipment should not impose limitations on the user freedom and, to meet such goal, low-cost is generically the best option.
Students must be able to experiment and learn without having to constantly worry about possible damage to expensive equipment.
