\subsection{Hardware} \label{sec:proposed-hardware}

While keeping a careful consideration of characteristics between options and maintaining the requirements in focus, hardware parts were chosen to build each section of the demonstrator.

\subsubsection{Master node}

Taking into account the two operation modes the demonstrator should have, we extrapolated that the master node must be able to perform numeric calculations and serve as an EtherCAT master device.
As the EtherCAT master implementation can be achieved using a generic Ethernet MAC interface card (refer to \ref{subsubsec:master_devices} for an explanation), everything the master node requires in term of hardware is a computational base (computer, microcontroller, etc.) with access to a generic Ethernet MAC interface card.

As of today, most education facilities provide students with access to desktop computers.
For many years now, motherboard vendors have integrated Ethernet MAC interface cards into the motherboards themselves, as it has become the de facto standard for Internet connectivity in desktop computers.

As such, we decided to implement the master device in a desktop computer in order to minimize costs and leverage the computational power modern computer systems possess.
