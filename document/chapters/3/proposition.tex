\section{Proposed architecture} \label{sec:proposed-arch}

By the end of the development phase, we created a system that fulfils all requirements described previously.
The system mimics a simplified DCS architecture by employing a simple master-slave configuration using an RTE network.

As explained in \autoref{subsec:local_remote}, using the local control topology, the master node will act as a motion controller, feeding the motor controller (the field device) with position/velocity references through the RTE network.
This one, in turn, will perform the necessary algorithm computations to achieve motor position/velocity control using the references provided by the master node and the feedback values it acquires from the motor's encoder.

When using the remote control topology, the master node will be responsible for all computations in the system.
The field device will act as a simple I/O interface to the motor, generating the waveform fed to the motor and acquiring and decoding the motor's speed and position.
The decoded position and velocity values are then sent to the master node as feedback variables and the motor output is received from it as an output variable.

Graphical representations of both these scenarios can be seen in \autoref{fig:local_control} and \autoref{fig:remote_control}, respectively. 
%\footnote{NOTE: These are placeholder images, I will replace them with better images or graphs designed by me}

\begin{figure}[htp]
	\centering
	\includegraphics[width=1\textwidth]{local_control.png}
	\caption{Graphical representation of the local control scheme (adapted from \cite{rte:motion-control-over-rte})}
	\label{fig:local_control}
\end{figure}

\begin{figure}[htp]
	\centering
	\includegraphics[width=1\textwidth]{remote_control.png}
	\caption{Graphical representation of the remote control scheme (adapted from \cite{rte:motion-control-over-rte})}
	\label{fig:remote_control}
\end{figure}


\section{Hardware development} \label{sec:hardware-devel}

% Motor Assembly
% `- Encoder soldering
% `- Cable soldering
% `- Magnetic disc attachment
\subsection{Motor assembly}
We began working on the hardware implementation by assembling the motor parts, which require some soldering.
First we soldered the encoder board on the motor terminals, being careful to leave enough motor shaft length available to be able to attach the encoder's magnetic disc on it.
Then we proceeded to solder a 6-wire cable to the encoder board terminals, which export all necessary electrical connections.
The exported connections are the encoder power supply ($VCC$ and $GND$, assigned to the blue/white-blue wire pair), the encoder output signals ($A$ and $B$, assigned to the green/white-green wire pair) and the motor power signals ($M1$ and $M2$, assigned to the orange/white-orange wire pair) and they can be visualized on \autoref{fig:encoder_board}.
Due to the development of the project having been done from home, for reasons of simplicity and availability of resources we used a generic 8-wire Ethernet UTP cable for the above mentioned electrical connections, as seen in \autoref{fig:motor_assembled}.
Finally we attached the magnetic disc on the motor shaft so the encoder can work as expected.

\begin{figure}[htp]
	\centering
	\includegraphics[width=0.8\textwidth]{IMG_2824_scaled.JPG}
	\caption{Motor encoder board connections}
	\label{fig:encoder_board}
\end{figure}

\begin{figure}[htp]
	\centering
	\includegraphics[width=0.8\textwidth]{IMG_2825_scaled.JPG}
	\caption{Encoder board and cables soldered to the motor}
	\label{fig:motor_assembled}
\end{figure}

% Motor Support
% `- 3D design
% `- 3D printing
% `- Adding the motor

% Raspberry Pi stack assembly
% `- Raspberry Pi
% `- GPIO exporter
%   `- encoder cables
% `- DFR0592
%   `- PSU cable
% `- netHAT 52-RTE


\subsection{Software} \label{sec:proposed-software}



