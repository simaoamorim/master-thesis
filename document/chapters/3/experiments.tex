\section{Conceptual experiments} \label{sec:experiments}

With the proposed architecture in mind, which was described in section \ref{sec:proposed-arch}, we have envisioned a generalized conceptual experiment to fulfil the main goal of demonstrating the effects of the network cycle time influence in control applications.
We have defined it in a generic way so that variations and extensions to the base idea can easily be developed.
This way the demonstrator does not focus on a single possible experiment but on a set of experiments that share the same foundation.

The first and most basic conceptual experiment we considered involves predefining a velocity curve over a certain amount of time and executing it using both available control modes: the local control and the remote control.
Each of these will generate a trace log of velocity points (in this case) measured during execution.

Naturally, the definition of the movement curve can be randomly generated or taken from any real-world example, whatever interests the user the most.
Independently of which control mode is going to be be executed, the predefined movement curve is to be stored in the master node by whatever means, either hard-coded into the control program or by using some form of data storage and interpretation.
Because the master node's software implementation will be left to the end-user's responsability, so will be the generation or interpretation of reference values from the predefined movement curve.
The only thing to be taken into account is that depending on which control mode is being used at a certain time, the slave device expects to receive different types of data: for local control the slave device expects to receive the control reference values (directly taken from the predefined movement curve) and for remote control it expects to receive the duty-cycle percentage to be applied to the motor (the actual 'output' value of the control loop).

As we aim to provide the means to export the recorded command and feedback values as a CSV file, we can now import them into any data processing software and perform relevant operations between the two traces.
The most simple comparison possible, and possibly the most effective, is to generate a graph that includes the two datasets simultaneously, which will create a visual representation of the system's performance in both control cases, making any differences in their behaviour quickly perceivable.

One variation that can be quickly derived from the afore mentioned one is to perform the same set of operations but while controlling and recording data regarding position control.
Position control systems are typically more sensible to performance diferences than velocity control ones, so although the experiment complexity increases slightly, more subtuble diferences in behaviour might be perceivable.
The increase in complexity is mostly due to the fact that position control implies velocity control so, in fact, instead of controlling one single property (velocity), we are now controlling two (position and velocity).
In this case, we may not care about the velocity control per se, but we still need to control it in order to control the position.


