\subsection{Error detection and diagnostics}

The first diagnosis feature present in EtherCAT is the ability for the master device to lookup the actual topology of the network segment it is present in, as referred in \autoref{subsec:ecat-topology}.
This provides a first insight on a possible cause for problems in the communication system, as a non-ideal network topology may prevent or limit the capabilities of the control system itself.

ESCs also have the ability to perform checksums on the fly, while processing the incoming packets.
Each EtherCAT packet contains a Working Counter field which is incremented each time the packet is processed on an addressed node.
If a checksum fails at any point, both the slave devices that are placed downstream of where this check failed and the master device are notified.
The latter will then discard such information, not forwarding it to the control application, and can request the Working Counter values from the slaves to try and identify where in the network the error was introduced.
This is a type of error identification which is impossible to be performed on a typical fieldbus network, as the physical medium is common to every single device o such networks (typical bus system).

If communication problems are suspected, network traffic sniffers such as the popular \emph{Wireshark} application can be utilized to visualize the actual information being transmitted on the network.
This is possible because EtherCAT transmits its frames transparently inside the Ethernet frame, as was shown in \autoref{fig:ecat-frame}.
