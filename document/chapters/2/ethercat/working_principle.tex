\subsection{Working principle} \label{subsec:ecat_principle}

The \emph{EtherCAT} protocol employs a master/slave configuration where only the former is allowed to initiate a data transfer.
Effectively this means the master node is responsible for maintaining periodic communication with all nodes.
The master device requires a simple \emph{Ethernet} \textbf Medium \textbf Access \textbf Controller ({\bfseries MAC}), meaning it can be implemented in virtually any device with a standard \emph{Ethernet} port, and programmed with any real-time operating system and software.
Slave devices require an \textbf \emph{EtherCAT} \textbf Slave \textbf Controller ({\bfseries ESC}) which processes the frames entirely in hardware.
This allows the network performance to be predictable and independent of specific slave device implementations.

The \emph{EtherCAT} frame sent by the master passes through each and every node on the network until it is sent back to it by the last node in each branch.
Because slave devices use a specialized communication chip, their data is inserted in the frame ``on the fly''.
This means that the master node can exchange data with every node on the network with a single \emph{EtherCAT} frame.
