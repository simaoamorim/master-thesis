\subsection{Working principle} \label{subsec:ecat_principle}

The \emph{EtherCAT} protocol employs a master/slave architecture.
Only the master device is allowed to initiate trasmissions and it fully controls the its slaves.
The commands issued by the master evoke responses from the slave devices.

The master node is responsible for maintaining periodic communication with all nodes.
It only requires a simple Ethernet Medium Access Controller (MAC), which means it can be implemented in virtually any device with a standard Ethernet port.

The communication is based on simple Ethernet telegrams encapsulating EtherCAT frames.
The telegrams are identified via the Ethertype field on the Ethernet header.
If the value of such Ethernet header field matches the EtherCAT identifier (\verb|0x88A4|), then the contents of this telegram will be treated as an EtherCAT telegram.

The EtherCAT frame is composed of an EtherCAT header and one or more EtherCAT commands.
Because EtherCAT frames can contain several commands, the master device can address several slaves using a single frame.
This improves the bandwidth utilization and makes a more efficient usage of the Ethernet frame.

EtherCAT networks typically employ an open ring topology, leaving the master Ethernet interface to be connected to one of the ends.
The master device issues the EtherCAT packets to the MAC address of the first slave device.

Slave devices receive a telegram, process it and forward it to the next slave device on the ring all at the same time.
This means EtherCAT slave devices process the telegrams on-the-fly.
If a certain slave device determines it is being addressed by an EtherCAT command, the data is retrieved and/or put into the frame on-the-fly, as the frame is traversing it.
After the last device on the ring has received and processed the frame, it gets sent back to the master device so it can process the responses to the commands it sent.

The frame processing and forwarding on the slave devices is entirely done in hardware.
As such, slave devices require a specilized Application Specific Integrated Circuit (ASIC) called an EtherCAT Slave Controller (ESC).
The ESCs are specialized in the described operations and only introduce a small and predictable processing delay for such operations, on the order of nanoseconds.
This allows the EtherCAT network segment performance to be predictable and independent of specific slave device implementations.

Each EtherCAT command contains a Working Counter (WC) field that is incremented each time an addressed slave prossesses the frame.
This allows the master device to determine if every addressed slave is actively communicating, although it does not guarantee data integrity.
For that purpose the standard Ethernet CRC is used to verify the message correctness \cite{technology:rte2}.
