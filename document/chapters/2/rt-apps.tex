\section{Real-time applications}

Modern automation systems have adopted a distributed computing architecture, as processing power is increasing every year and, consequently, becoming cheaper.
Real-Time control applications don't necessarily require fast execution, but it means the process itself depends on the passage of time to be correctly executed. \cite{technology:rte}

These systems don't depend only on the data that is gathered on the plant-floor, but also on when it is acquired.
In the robotics field, for instance, it's important to have a solid and short control period for motion applications, but it's even more important to make sure the sensor data used to perform the calculations is the most recent possible.

Now, depending on the application itself, it may have stricter or looser timing requirements.
As such, they are classified as hard real-time (HRT) or soft real-time (SRT) applications.

\begin{itemize}
	\item \textbf{Hard Real-Time}: These systems have the strictest set of requirements in terms of timeliness, meaning the incorrect operation of such systems may cause a catastrophic failure of the entire process, with the possibility of putting lives in danger. Systems such as self driving cars or medical-grade robotics have such requirements.

	\item \textbf{Soft Real-Time}: These systems have much leaner requirements as incorrect operation due to failed deadlines, even though not desirable, does not mean a loss of life or equipment. A couple of examples are GPS systems or radio broadcast systems where delays may not be desirable but they are also not critical.
\end{itemize}