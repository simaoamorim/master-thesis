\section{Real-time Ethernet networks}\label{sec:rt-networks}

As industrial automation systems have been moving towards decentralized processing architectures, lean coupling methods between the different parts have started to evolve.
As the industry demands for a single network type to fulfil the needs across all hierarchy levels (from plant-floor to boardroom), Industrial Ethernet systems started to gain weight on the field.

As most applications that control industrial processes have some sort of real-time requirements, the introduction of an Ethernet network into the control system needs not to disrupt its real-time characteristics.
As such, Real-Time Industrial Ethernet networks have strict timing requirements which makes the usage of common Ethernet not adequate, because the IEEE 802.3 standard, as is, does not guarantee a deterministic timing for delivery of message packets.

A few adaptations of this standard have surfaced over the years, focusing mostly on providing such deterministic delivery.
Some implementations categorize data packets based on their priority relative to the control system so that, at least the ones that are critical, are delivered as fast as possible.

% General Characteristics
Commercial Ethernet systems are based on a 7-layer encapsulating scheme named OSI (Open Systems Interconnection).
Adaptations such as Ethernet/IP or Profinet/IO make use of this scheme and simply manage on which layers message packets are sent, depending on their priority, or try to implement some prioritization mechanism for certain data packets.
On the other hand, implementations such as EtherCAT drop most of the OSI model and create a bare scheme on top of the Link layer (the actual Ethernet protocol).
