\section{Real-time Ethernet networks}\label{sec:rt-networks}

% What are they
As explained in \autoref{sec:context}, real-time \emph{Ethernet} networks have strict timing requirements which makes the usage of common \emph{Ethernet} not adequate.
The IEEE 802.3 standard, as is, does not guarantee a deterministic timing for the delivery of message packets.
As such, a few adaptations of this standard have surfaced over the years, focusing mostly on providing this deterministic delivery of, at least, the message packets that are critical for the system.

% General Characteristics
Commercial \emph{Ethernet} is based upon a 7-layer addressing scheme named \textbf{OSI} (\textbf Open \textbf Systems \textbf Interconnection).
Adaptations such as \emph{Ethernet/IP} or \emph{Profinet/IO} still make use of this 7-layer scheme but manage on which layer message packets are sent, depending on their priority.
On the other hand, implementations such as \emph{EtherCAT} drop most of the \emph{OSI} model and create an entirely new addressing scheme based on the model's second layer.

\subsection{EtherCAT} \label{sec:ethercat}

\subsection{Working principle} \label{subsec:ecat_principle}

The \emph{EtherCAT} protocol employs a master/slave architecture.
Only the master device is allowed to initiate trasmissions and it fully controls the its slaves.
The commands issued by the master evoke responses from the slave devices.

The master node is responsible for maintaining periodic communication with all nodes.
It only requires a simple Ethernet Medium Access Controller (MAC), which means it can be implemented in virtually any device with a standard Ethernet port.

The communication is based on simple Ethernet telegrams encapsulating EtherCAT frames.
The telegrams are identified via the Ethertype field on the Ethernet header.
If the value of such Ethernet header field matches the EtherCAT identifier (\verb|0x88A4|), then the contents of this telegram will be treated as an EtherCAT telegram.

The EtherCAT frame is composed of an EtherCAT header and one or more EtherCAT commands.
Because EtherCAT frames can contain several commands, the master device can address several slaves using a single frame.
This improves the bandwidth utilization and makes a more efficient usage of the Ethernet frame.

EtherCAT networks typically employ an open ring topology, leaving the master Ethernet interface to be connected to one of the ends.
The master device issues the EtherCAT packets to the MAC address of the first slave device.

Slave devices receive a telegram, process it and forward it to the next slave device on the ring all at the same time.
This means EtherCAT slave devices process the telegrams on-the-fly.
If a certain slave device determines it is being addressed by an EtherCAT command, the data is retrieved and/or put into the frame on-the-fly, as the frame is traversing it.
After the last device on the ring has received and processed the frame, it gets sent back to the master device so it can process the responses to the commands it sent.

The frame processing and forwarding on the slave devices is entirely done in hardware.
As such, slave devices require a specilized Application Specific Integrated Circuit (ASIC) called an EtherCAT Slave Controller (ESC).
The ESCs are specialized in the described operations and only introduce a small and predictable processing delay for such operations, on the order of nanoseconds.
This allows the EtherCAT network segment performance to be predictable and independent of specific slave device implementations.

Each EtherCAT command contains a Working Counter (WC) field that is incremented each time an addressed slave prossesses the frame.
This allows the master device to determine if every addressed slave is actively communicating, although it does not guarantee data integrity.
For that purpose the standard Ethernet CRC is used to verify the message correctness \cite{technology:rte2}.


\subsection{The protocol}

The \emph{EtherCAT} protocol embeds its own frames into a standard \emph{Ethernet} frame, signing it with an hexadecimal value of $\mathtt{0x88A4}$ on the \emph{Ethernet}'s type header field.
Other protocol stacks like TCP/IP or UDP/IP can be used concurrently with \emph{EtherCAT}, but they are not required.
These are encapsulated into a separate mailbox so they do not disrupt real-time process data transmissions.
The fact that this network does not use these stacks means it has lower communication overhead.

The \emph{EtherCAT} frames are, themselves, divided into several datagrams, as show in \autoref{fig:ecat-frame}.
These can be addressed to specific devices using their node address or be sent to multiple devices, concurrently, using a logical address.
The datagram header contains information about the type of operation to perform, which can be one of three options: read, write of concurrent read-write operations.

\begin{figure}[htp]
	\centering
	\includegraphics[width=1\textwidth]{EtherCAT_Technology_01_Protocol.jpg}
	\caption{EtherCAT frame structure \cite{protocol:ethercat}}
	\label{fig:ecat-frame}
\end{figure}

Datagrams include all information regarding data access which permit the master device to decide what data to access and when, meaning a fixed process data structure is not required.
Effectively, master devices can update variables with different cycle times, possibly relieving some processing power.
As an example, for a system that requires motion control, the motor drives can get their parameters updated with a 1ms period, while discrete Inputs/Outputs (I/Os) can be updated with a 20ms period (typical control applications).

Each slave contains a unique node address which is assigned during network configuration.
Because node addresses are static, they can be used to target the specific node, even if the underlying network topology changes.
In addition, slaves can also be addressed by their location on the network, but this is usually only used during network initialisation to check for topology changes.
This is done by comparing a configured list of node addresses and their location on the network with the discovered topology.

On system initialisation, multiple logical addresses can be configured on each node, allowing a single datagram to target multiple physical devices.
The cyclical exchange of process information uses logical addressing to execute the data transfers.

This type of addressing scheme also allows slave-to-slave communication.
There are two possibilities of achieving this:
\begin{enumerate}
	\item If the process structure is constant, sending data to another slave which is further downstream can be done in the same bus cycle;
	\item If the process is not constant or the network has a dynamic topology, slave-to-slave communication can go through the master device and, because of \emph{EtherCAT}'s performance, this is still faster than other traditional communication stacks (TCP/IP, UDP/IP, etc.).
\end{enumerate}

EtherCAT can also benefit from the modern system's \textbf Direct \textbf Memory \textbf Access (DMA) feature, which removes the necessity for a CPU to explicitly transfer data from physical RAM to a peripheral device.
This means that a master device application only needs to construct the EtherCAT frame and place it on a specific memory region, leaving the DMA controller to actually pass the data over to the Ethernet MAC controller, saving CPU for the actual data processing.


\subsubsection{Topology} \label{subsec:ecat-topology}

\emph{EtherCAT} supports a variety of network topologies like \emph{line}, \emph{tree}, \emph{star} or \emph{daisy-chain}.
Many ESCs and I/O modules already include ports to create network branches, which eliminates the need to use switches or any other type of infrastructure components.
Regardless, classical \emph{Ethernet} star topology can be used to implement an \emph{EtherCAT} network.
When designing a certain network, multiple topologies can be combined into a hybrid topology network.
\autoref{fig:ecat-topology} presents a possible illustration of such case.

\begin{figure}[htp]
	\centering
	\includegraphics[width=\textwidth]{EtherCAT_Technology_03_Topology.jpg}
	\caption{Example of a hybrid topology EtherCAT network \cite{protocol:ethercat}}
	\label{fig:ecat-topology}
\end{figure}

ESCs also include support for a ``Hot Connect'' feature which means existing nodes can be removed and new nodes can be added to the network during runtime.
The controllers can detect these changes in a very short time (typically less than 15$\mu$s), allowing a smooth state transition without interfering with the rest of the network.

There is also a big flexibility in terms of available cabling option, from inexpensive industrial \emph{Ethernet} cables to fiber optics, having the entire Ethernet wiring possibilities available for use.

EtherCAT gateways provide the means to incorporate other fieldbus networks as a subnetwork.
This allows a gradual changeover between fieldbuses by keeping network sections that may contain components which still do not support the EtherCAT interface.

Due to the fact that EtherCAT uses a 16-bit address length, up to $65535$ devices can exists in a single network segment, which makes scalability virtually unlimited.
This large device count removes the need to use bus extension methods, like traditional gateways, providing even the largest EtherCAT networks the best possible performance without delays.


\subsubsection{Distributed clocks}

Certain types of control applications require simultaneous actions to be taken.
In the robotics field, for instance, movement control implies that several servo controllers synchronize their actions in order to  achieve the desired speed or position path.
In a DCS, it is common for these actions to span multiple nodes on the network.
Therefore, these nodes require some sort of subsystem that is capable of guaranteeing action synchronicity between them.
Synchronous communication protocols already tackle this problem in an implicit fashion, but EtherCAT needs an explicit solution: distributed clocks (DC).

Every ESC contains a highly precise clock source in its design, as well as a purely hardware based calibration system.
The first slave DC in an EtherCAT network is used as a reference value, being distributed to all other slave nodes.
This way, all these clocks present on the network are adjusted to the same reference value, allowing hardware propagation times to be calculated and taken into account on the calibration process.
This can either be done during network initialization or continuously throughout the operational period.

This distributed clock technology has been proven to introduce much less jitter on the communication system, when compared to synchronous protocols, with common values below the microsecond mark.
Very precise output updates and very accurate timestamping on the input values are achieved with this implementation.
It is a very important feature needed by the aforementioned movement control systems, as these rely heavily on precise input timestamping to accurately calculate velocities, as these are usually derived from position measurements.
These systems also require the position measurements to be taken in periodic intervals, with as little jitter as possible.
The distributed clocks also factor in on this topic, as they can generate much more accurate triggers than the network bus itself.

In addition, this technology removes the ensuring of actions's synchronicity between slave nodes from the scope of the master device.
In fact, the local clocks can be utilized to trigger actions on the slave nodes, such as updating outputs and reading inputs.
Consequently, the master's EtherCAT communication stack can be implemented entirely in software or on simple Ethernet hardware because the master's stack jitter becomes practically irrelevant, for as long as the EtherCAT datagram is sent early enough to reach the slave device before its local clock triggers the relevant action.




\subsection{Ethernet/IP} \label{sec:ethernetip}




\subsection{Profinet/IO} \label{sec:profinetio}



