\section{Real-time Ethernet networks}\label{sec:rt-networks}

As industrial automation systems have been moving towards decentralized processing architectures, lean coupling methods between the different parts have started to evolve.
As the industry demands for a single network type to fulfil the needs across all hierarchy levels (from plant-floor to boardroom), Industrial Ethernet systems started to gain popularity on the industrial automation world.

As most applications that control industrial processes have some sort of real-time requirements, the introduction of an Ethernet network into the control system needs not to disrupt its real-time characteristics.
As such, Real-Time Industrial Ethernet networks have strict timing requirements which makes the usage of common Ethernet not adequate, because the IEEE 802.3 standard, as is, does not guarantee a deterministic timing for delivery of message packets.

A few adaptations of this standard have surfaced over the years, focusing mostly on providing such deterministic delivery.
Some implementations categorize data packets based on their priority relative to the control system so that, at least the ones that are critical, are delivered as fast as possible.

% General Characteristics
Commercial Ethernet systems are based on a 7-layer encapsulating scheme named OSI (Open Systems Interconnection).
Several standard approaches have been defined for how to achieve deterministic communication.
Three different approaches have been explored in depth by describing the key concepts on their basis.
These were named `Top of TCP/IP', `Top of Ethernet' and `Modified Ethernet' \cite{rte:rte-for-automation}, according to the approach each made to solve the deterministic delievery of messages.

The first category implements mechanisms on top of the TCP/IP protocol stack, without any modifications applied to it.
Networks using this approach communicate trasparently over over network boundaries, even through routers.
It is therefore possible to create networked automation systems anywhere on the world, just like Internet technology.
The biggest disadvantage is the necessary processing power and memory, that still introduce nondeterministic delays \cite{rte:rte-for-automation}.
Industrial networks such as Modbus/TCP and EtherNet/IP \cite{protocol:ethernetip} adopted this type of approach for realizing deterministic communication.

The second category of RTE networks called `On top of Ethernet' do not use the protocol stacks that the previous category is based upon.
Instead, these networks implement a dedicated protocol stack on top of the Ethernet frame, and each one of them specifies a different \verb|Ethertype| field.
Such protocols can coexists with standard IP stacks and some even use them \cite{rte:rte-for-automation}.
Examples of solutions following this approach are the PROFINET CBA and Ethernet for Plant Automation.

The third category, unlike the previous two, provide a modified Ethernet protocol or hardware.
As traditional Ethernet employs a typical star topology, devices need to be connected to a central switch.
Fieldbuses replaced this topology with simpler ones, such as bus or ring, in order to reduce cabling costs.
As RTE networks usually replace older automation fieldbuses, solutions should allow the usage of such simpler cabling methods.
As a result, it becomes mandatory for solutions that aim for hard real-time performance to employ hardware modifications on the devices or network infrastructure.
Employing bus and ring topologies for these networks mandates that each field device integrates switching capabilities.
Nonetheless, these modifications allow non real-time traffic to be transmitted without modifications \cite{rte:rte-for-automation}.
Networks such as SERCOS, EtherCAT \cite{protocol:ethercat} and PROFINET IO \cite{protocol:profinet} follow this approach.
