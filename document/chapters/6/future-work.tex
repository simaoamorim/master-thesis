\section{Future work} \label{sec:future-work}
The developed system includes some limitations that could be enhanced in future works.
Some have already been presented on previous chapters, but we will also include them in this section.

% `- Improve the encoder IO and velocity/position calculation methods
One of the first improvements that can be implemented is a better algorithm for computing the speed and position of the motor on the slave device.
A revised approach based on callback mechanisms and timestamping could greatly increase the precision on such measurements.
The GPIO interface library in use (\verb|libgpiod|) includes functions to register callbacks when input signals change.
The documentation on such functions is very limited and a deeper exploration of the concept is required.

% `- Improve the software's user friendliness
As summarized previously, the developed system in intended to serve a proof-of-concept purpose.
As such, the developed software focused solely on the necessary technical functionality.
User interaction has not been considered when implementing the software, so future works could work on a more pleasant user experience when using this system.

% `- Use a PREEMPT-RT kernel on the Raspberry Pi
More accurate results can also be obtained by making sure the Linux kernel in use supports running real-time applications.
As we have explained previously, we are using a stripped down version of the Raspberry Pi OS in order to minimise software bloating.
Nonetheless, some patches exist for the Linux kernel that make it more suitable for real-time applications.
At the time of writing, works are still being carried out to include one such patch onto the kernel itself: the Preempt-RT patches.
Unfortunately the Raspberry Pi OS does not provide a kernel version with such patches (unlike Debian, its parent).
As such, future works could obtain more deterministic software periods using a kernel with real-time capabilities and, consequently, improve the data accuracy.
