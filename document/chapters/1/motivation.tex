\section{Motivation} \label{sec:motivation}

As modern control systems evolve in complexity, providing automation students with adequate training on the technologies used today is vital to ensure their integration on the industry is as smooth as possible.
Learning what benefits and disadvantages a certain component or system brings to the automation world allows the future engineers to make more informed decisions when designing new control systems.

With industrial automation systems continuously improving their connectivity to the digitized world, it becomes essential to give students first-hand contact with the technology.
Keeping this purpose in mind, the course on Electrical and Computers Engineering at \Feup{} is expanding its curricular suite with a class on real-time industrial communication networks.
As real-time Ethernet networks have dominated the market for some years now, these will be the main focus of the class.

In order to provide these students with the best possible experience, a practical demonstrator focused on education about the pros and cons of industrial real-time Ethernet networks needs to be developed.
Solutions on the market usually target end customers who are already considering adopting the technology on their plant-floors.
As such, manufacturers tend to restrict their demonstrators to very specific scenarios where the technology works best, is pre-configured and also intentionally leaves out any disadvantages that may exist on a generic approach.

With this mind, students can learn more about the features, limitations and how to implement and configure an industrial real-time Ethernet network by actually needing to do so on a practical approach.
So, a demonstrator which needs to be configured, have its implementation completed, to a certain extent, and allows several experiments to be executed when fully implemented seems to fit the purpose of providing a good tool for first-hand contact with industrial real-time Ethernet networks.
