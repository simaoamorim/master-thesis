\section{Objectives} \label{sec:objectives}
% * solid foundation for future works
% ** hardware accessibility
% ** hardware interchangeability
% ** software modularity
% ** solid documentation
% * network cycle time demonstrator
% ** local control
% ** remote control
% ** position control
% ** velocity control

This thesis mainly intends to produce a solid foundation for practical demonstrators of industrial real-time Ethernet networks.
The work developed on this dissertation will build upon the concept of network cycle time, known as a periodic deadline for the sequential delivery of data packets on the network.
The solution should provide the user a practical and reality-based experimentation tool to observe the effects of network cycle time in control applications.
Because we intend to create a good foundation for more advanced demonstrators, we aim at providing a robust and well documented solution with a high level of reusability and adaptability.

When automating industrial processes, especially in the robotics department, some sort of movement control is usually implied.
The most common are velocity and position control, so our demonstrator will focus on these while simultaneously using a real-time Ethernet network to communicate with a remote controller.
This remote controller will provide either the control set-points or the control algorithm itself.

With that being said, we aim at creating a demonstrator system comprised of 2 nodes, connected through a real-time Ethernet network.
The slave device (known as a field device in DCSs) will be the main focus of our development and will provide a motor control interface and an incremental quadrature encoder interface.
The combination of these two interfaces will allow us to perform control of the motor's position or velocity using a simple PID controller.
This field device will have two modes of operation:

\begin{itemize}
	\item Local control: the control loop will be closed on the field device itself by means of a simple PID control algorithm, receiving set-points from the real-time Ethernet network. This type of control is expected to barely be impacted by the network performance.

	\item Remote control: the field device will act as a simple remote I/O card, synchronising the physical inputs and outputs with values being transmitted on the real-time Ethernet network. This will effectively mean the control loop will be closed on the master device, which will run the control algorithm, making it very susceptible to the effects of network cycle time.
\end{itemize}

These configurations will allow the user to create a base dataset from the local control and compare the data generated by the remote configuration with the local control dataset, while using different values for the network cycle time.
