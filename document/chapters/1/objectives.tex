\section{Objectives} \label{sec:objectives}
% * solid foundation for future works
% ** hardware accessibility
% ** hardware interchangeability
% ** software modularity
% ** solid documentation
% * network cycle time demonstrator
% ** local control
% ** remote control
% ** position control
% ** velocity control

This thesis mainly intends to produce a solid foundation for practical demonstrators of industrial real-time networks. The work developed on this dissertation will build upon the concept of network cycle time, providing whoever interacts with the proposed solution a practical and reality-based experiment to observe the effects of network cycle time in control applications. Because we intend to create a good foundation for more advanced demonstrators, we aim at providing a robust and well documented solution with a high level of reusability and adaptability.

When automating industrial processes, some sort of movement control is usually implied. The most common are velocity and position control, so these will be considered for this demonstrator.

In order to focus on network cycle time, we will develop a \emph{field device} capable of running two configurations:

\begin{itemize}
	\item Local control: the control loop will be closed on the \emph{field device} itself by means of a simple and fast algorithm which receives set-points from the real-time network. This type of control is expected to barely be impacted by the network performance.

	\item Remote control: the \emph{field device} will act as a simple remote I/O card, synchronising the physical inputs and outputs with values being transmitted on the real-time network. This will effectively mean the control loop will be closed on some other \emph{process control} device (such as a \emph{PLC}), making it very susceptible to the effects of network cycle time.
\end{itemize}

These configurations will allow the user to create a base dataset from the local control and compare the data generated by the remote configuration with the local control dataset.
