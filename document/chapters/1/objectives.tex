\section{Objectives} \label{sec:objectives}
% * solid foundation for future works
% ** hardware accessibility
% ** hardware interchangeability
% ** software modularity
% ** solid documentation
% * network cycle time demonstrator
% ** local control
% ** remote control
% ** position control
% ** velocity control

This thesis mainly intends to produce a solid foundation for practical demonstrators of industrial real-time networks. The work developed on this dissertation will build upon the concept of network cycle time, providing whoever interacts with the proposed solution a practical and reality-based experiment to observe the effects of network cycle time in control application. Because we intend to create a good foundation for different and more advanced demonstrators to be built upon, we aim at providing a robust and well documented solution with a high level of reusability.

In order to provide a network cycle time demonstrator, we will develop a networked \emph{field device} capable of running in two configurations:

\begin{itemize}
	\item Local control: the control loop will be closed on the \emph{field device} itself by means of a simple and fast algorithm which receives the control set-points from the real-time network.
	\item Remote control: the \emph{field device} will work as a simple remote I/O card which will synchronise the physical inputs and outputs with values being transmitted on the real-time network. This will effectively mean the control loop will only be closed on the \emph{process control} device, having the real-time network operate inside this control loop.
\end{itemize}

These configurations allow the user to create a base dataset from the local control configuration and compare the dataset generated by the remote control configuration with that base.