\chapter{Introdução} \label{chap:intro}



\section{Contexto} \label{sec:contexto}

- Curso pretende adicionar à oferta formativa uma parte sobre redes de comunicacao industriais de tempo real;
- EtherCAT é uma dessas redes e contém um leque abrangente de possibilidades de utilização: Controlo de processos automatizados, controlo de manipuladores robóticos, etc.


\section{Motivação} \label{sec:motivacao}

- De modo a fornecer um entendimento mais fácil acerca das capacidades e limitações da rede EtherCAT, faz sentido desenvolver um demostrador prático que possa ser utilizado para esse efeito
% \blindtext


\section{Objetivos} \label{sec:objetivos}
Estudar em detalhe o funcionamento da rede EtherCAT:

\begin{itemize}
\item Quantidade de dados de transmissao
\item Tipologia de redes suportadas:
    \subitem ``'drop lines, lines, daisy chains or tree structures`'
    \subitem Métodos de difusão/transmissão dos dados
    - Resposta temporal
    - Sincronização de relógios:
        - Distributed Clock
        - (...)
    - Hardware necessário
- Estudar possíveis demonstrações:
- 
\end{itemize}



% \Blindtext




