%% artigo-exemplo.tex

\documentclass[a4paper]{IEEEtran}
\usepackage[utf8]{inputenc}
\usepackage{latin}
\markboth{Provas de Dissertação do MIEEC---Julho de 2008}{}
\usepackage[portuges]{babel}

\ifCLASSINFOpdf
  \usepackage[pdftex]{graphicx}  
\else
  \usepackage[dvips]{graphicx}
\fi

\renewcommand{\footnoterule}{\noindent\rule{0.5\columnwidth}{0.5pt}\vspace*{3pt}}

\begin{document}

% Título (usar \\ para quebra de linha)
\title{Prepara��o de artigos para as\\
  Actas das Provas de Disserta��o do MIEEC}


% author names and affiliations
% use a multiple column layout for up to three different
% affiliations
\author{\IEEEauthorblockN{Nome Autor$^*$}%
\thanks{$^*$Contacto autor}
\IEEEauthorblockN{2� Autor (orientador)$^\dag$}%
\thanks{$^\dag$Contacto orientador}
\IEEEauthorblockN{3� Autor (co-orientador)$^\ddag$}%
\thanks{$^\ddag$Contacto co-orientador}
}
% make the title area
\maketitle

% \markboth{Uma parte}{Outra parte}

\begin{abstract}
  Modelo de formatação do artigo de uma folha sobre o trabalho de
  dissertação ou de projecto realizado no âmbito da unidade curricular
  Dissertação do MIEEC. Não inserir referências no sumário. Não apagar
  a linha em branco acima do sumário.
\end{abstract}

\begin{IEEEkeywords}
Três ou quarto palavras-chave em ordem alfabética.
\end{IEEEkeywords}

\section{Introduction}

\IEEEPARstart{T}{his} document is based on the guidelines for
preparing IEEE Transactions and Journal papers~\cite{ieee07}. Do not
change the font sizes or line spacing to squeeze more text into a
limited number of pages. Use italics for emphasis; do not underline.

To insert images in Word, position the cursor at the insertion point
and either use Insert | Picture | From File or copy the image to the
Windows clipboard and then Edit | Paste Special | Picture (with
``float over text'' unchecked).


\section{Procedimento para submissão de um artigo}

\subsection{Entrega}

The paper should be sent to jms@fe.up.pt before the 31st of July 2008.

\subsection{Segunda sub-secção}


O artigo pode ser escrito em português ou em inglês. Todos os artigos
serão compilados e editados no livro de Actas das Provas de
Dissertação do MIEEC.


\section{Formatação}

\subsection{Equações}

Se usar o Word, procure usar o editor de equações Microsoft Equation
Editor. As equações devem ser numeradas consecutivamente com o número
entre parênteses na margem direita, como em (\ref{eq:1}).

\begin{equation}
  \label{eq:1}  
  \alpha = \int_0^\pi F(r,\sigma) \cdot d\sigma
\end{equation}


\subsection{Figuras}

Deve ser inserida uma (e apenas uma) figura no artigo -- aquela que se
considere ser a mais representativa do trabalho realizado. Deve ser de
boa qualidade, a cores ou a preto e branco, num formato que permita a
conversão para pdf do artigo.

\begin{figure}[ht]
  \centering
  \includegraphics[width=.8\linewidth]{tux}
  \caption{Incluir apenas uma figura no artigo.}
  \label{fig:1}
\end{figure}


\subsection{Referências}

As referências devem ser incluídas tal como já apresentado no modelo
de formatação das dissertações \cite{pub08}.

\subsection{Tabela}

Pode ser inserida apenas uma tabela no artigo. 

\begin{table}[h]
\caption{Exemplo de tabela}

\begin{tabular}{|c|ccc|r|}
  \hline
$k$ &  $x_1^k$    &   $x_2^k$  & $x_3^k$   & remarks  \\
        \hline
0   & -0.3 & 0.6 & 0.7  &  \\
1   & 0.47102965 & 0.04883157 & -0.53345964  & *\\
2   & 0.49988691 & 0.00228830 & -0.52246185 & $s_3$ \\
3   & 0.49999976 & 0.00005380 & -0.52365600  & \\
4   & 0.5 & 0.00000307 & -0.52359743  & $\epsilon < 10^{-5}$ \\
7   & 0.5 & 0 & -0.52359878  & $\epsilon < \xi $ \\
        \hline
\end{tabular}
\end{table}


\section{Conclusion}

Resumir os principais resultados atingidos e identificar possíveis
desenvolvimentos futuros.

\section*{Agradecimentos}

Sem exagerar.

% referências

\bibliographystyle{IEEEtran}
\bibliography{refs}

\end{document}


